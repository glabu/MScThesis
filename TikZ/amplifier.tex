% !TeX program = lualatex
% Non-inverting operational amplifier with european standards
% Autor: Ismael Torres García
\documentclass[border=3pt]{standalone}
\usepackage[math-style=ISO,bold-style=ISO]{unicode-math}
\setmainfont[Renderer=OpenType]{Minion Pro}          %% Fuente comercial MinionPro de Adobe para el texto
\setmathfont[                                        %% Fuente MinionMath de Johannes Küsner para las matemáticas
    Extension = .otf,
    Path = /Users/ytoga/Library/Fonts/,
    Scale = 1,
    Script = Math,
    SizeFeatures = {
        {Size = -6, Font = MinionMath-Tiny,
        Style = MathScriptScript},
        {Size = 6-8.4, Font = MinionMath-Capt,
        Style = MathScript},
        {Size = 8.4-, Font = MinionMath-Regular},
    },
]{MinionMath-Regular}
\setmathfont[                         
    Extension = .otf,
    Path = /Users/ytoga/Library/Fonts/,
    version = bold,
]{MinionMath-Bold}

\usepackage[european]{circuitikz}
\ctikzset{
monopoles/vcc/arrow={Triangle[width=0.8*\scaledwidth, length=\scaledwidth]},
monopoles/vee/arrow={Triangle[width=6pt, length=8pt]},
}
\usepackage{siunitx}
%%%%%%%%%%%%%%%%%%%%%%%%%%
%%% COMIENZO DEL DOCUMENTO
%%%%%%%%%%%%%%%%%%%%%%%%%%
\begin{document}
\tikzset{every picture/.style={line width=0.2mm}}

\begin{circuitikz}	

%% CIRCUITO
\node[op amp] at (0,0) (opamp) {};
\node[eground] at (-4.5,-0.5) (ground) {};
\draw (opamp.-) to[R, l_=$R_1$, name=R1] ++(-3.3,0) -- (ground);
\draw (opamp.+) to[short,-o] ++(-1.2,0) node[left] {$U_e$};
\draw ($(opamp.-)+(-0.45,0)$) to[short,*-] ++(0,2.3) to[R, l=$R_2$, i>=$i$, name=R2] ++(3.3,0) to [short, -*] ++(0,-2.79);
\draw (opamp.out) to[short,-o] ++(1.5,0) node[right] {$U_s$};

%% SATURACIÓN
\draw (opamp.up) -- ++(0,0.5) node[vcc] {$+U_{cc}$};
\draw (opamp.down) -- ++(0,-0.5) node[vee] {$-U_{cc}$};

%% CORRIENTE ELÉCTRICA
\draw[-{Triangle[round]}] ($(opamp.-)+(-0.15,0)$) -- ++(0.2,0) 
node[above,xshift={-2pt}] {\footnotesize $\qty{0}{A}$};
\draw[-{Triangle[round]}] ($(opamp.+)+(-0.15,0)$) -- ++(0.2,0) 
node[above,xshift={-2pt}] {\footnotesize $\qty{0}{A}$};
\draw[-{Triangle[round]}] (ground) -- ++(0,0.5) node[right] {$i$};

%% TEXTO
\node[shift={(-0.4,-0.27)}] at (opamp.-) {\footnotesize $U_-=U_+$};
\node[shift={(-0.4,-0.27)}] at (opamp.+) {\footnotesize 
$U_+=U_e$};

\end{circuitikz}
\end{document}
