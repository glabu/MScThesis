% !TeX program = lualatex
% Schematic of laser components
% Autor: Ismael Torres García
\documentclass[tikz,border=3pt]{standalone}
\usetikzlibrary{calc, arrows.meta, decorations.pathmorphing, decorations.markings}
\usepackage[math-style=ISO,bold-style=ISO]{unicode-math}
\setmainfont[Renderer=OpenType]{Minion Pro}          %% Fuente comercial MinionPro de Adobe para el texto
\setmathfont[                                        %% Fuente MinionMath de Johannes Küsner para las matemáticas
    Extension = .otf,
    Path = /Users/ytoga/Library/Fonts/,
    Scale = 1,
    Script = Math,
    SizeFeatures = {
        {Size = -6, Font = MinionMath-Tiny,
        Style = MathScriptScript},
        {Size = 6-8.4, Font = MinionMath-Capt,
        Style = MathScript},
        {Size = 8.4-, Font = MinionMath-Regular},
    },
]{MinionMath-Regular}
\setmathfont[                         
    Extension = .otf,
    Path = /Users/ytoga/Library/Fonts/,
    version = bold,
]{MinionMath-Bold}

\usepackage{xcolor}
\definecolor{azulito}{RGB}{0,255,255}
\definecolor{verdecito}{RGB}{124,252,0}
%%%%%%%%%%%%%%%%%%%%%%%%%%
%%% COMIENZO DEL DOCUMENTO
%%%%%%%%%%%%%%%%%%%%%%%%%%

\begin{document}

\tikzset{arrow/.style={blue,-{Stealth[length=1mm,width=1mm]},decorate,decoration={snake,amplitude=.4mm,segment length=1mm,post length=1mm}},
}

\begin{tikzpicture}

%% DEFINICIONES
\def\L{5} % Longitud de la cavidad
\def\H{3} % Altura de la cavidad
\def\h{4} % Separación vertical de la flecha
\def\l{1} % separación horizontal de la flecha
\def\angs{40}  % ángulo de salida intermedio
\def\ange{140} % ángulo de entrada intermedio
\def\angini{0} % ángulo de salida inicial con la base de flecha
\def\angfin{180} % ángulo de entrada final con la punta de flecha
\def\coorfin{10} % etiqueta del último vértice de la flecha

%% COORDENADAS
\coordinate (M1) at (-0.45*\L,0.05*\H);  % Esquina inferior izquierda medio
\coordinate (M2) at (0.45*\L,0.95*\H);   % Esquina inferior derecha medio
\coordinate (M3) at (-0.45*\L,-0.2*\H);  % Esquina inferior izquierda medio
\coordinate (M4) at (0.45*\L,-0.05*\H);  % Esquina inferior derecha medio
\coordinate (S) at (2.5*\l,0);           % El desplazamiento de la flecha
 
%% ESPEJOS
\draw (-0.5*\L,0) -- (-0.5*\L,\H) node[above,scale=0.5] {Espejo $100\%$ reflectante};
\draw (0.5*\L,0) -- (0.5*\L,\H) node[above,scale=0.5] {Espejo $90\%$ reflectante}; 

\foreach \x in {1,2,...,19}
    \draw (-0.5*\L,0.05*\H*\x) -- (-0.55*\L,{0.05*\H*(\x+1)});

%% MEDIO
\draw[fill=gray!10,rounded corners=1mm] (M1) rectangle (M2);
\node[align=center,scale=0.5] at ($0.5*(M1)+0.5*(M2)$) {Medio transparente o celda con átomos,\\iones o moléculas};

%% RADIACIÓN
\begin{scope}[shift=($0.5*(M1)+0.5*(M2)+(S)$)]
    \foreach \y in {1,2,...,\coorfin}{
        \coordinate (A\y) at (\l*\y/7,0.1*\h);
        \coordinate (B\y) at (\l*\y/7,-0.1*\h);
    }
    \coordinate (V) at ($0.5*(A\coorfin)+0.5*(B\coorfin)+(\l/2,0)$);
    % La flecha. Cambia \coorfin y añade más nodos A y B para hacerla más larga
    \draw[fill=red!70] (A1) to [out=\angini,in=\ange] (A2) to [out=-\angs,in=-\ange] (A3) to [out=\angs,in=\ange] (A4) to [out=-\angs,in=-\ange] (A5) to [out=\angs,in=\ange] (A6) to [out=-\angs,in=-\ange] (A7) to [out=\angs,in=\ange] (A8) to [out=-\angs,in=-\ange] (A9) to [out=\angs,in=\angfin] (A10) --++ (0,0.05*\h) -- (V) -- ($(B10)+(0,-0.05*\h)$) -- (B10) to [out=\angfin,in=\angs] (B9) to [out=-\ange,in=-\angs] (B8) to [out=\ange,in=\angs] (B7) to [out=-\ange,in=-\angs] (B6) to [out=\ange,in=\angs] (B5) to [out=-\ange,in=-\angs] (B4) to [out=\ange,in=\angs] (B3) to [out=-\ange,in=-\angs] (B2) to [out=\ange,in=\angini] (B1) -- cycle;
    \node[scale=0.5] at ($0.5*(A1)+0.5*(B10)$) {Salida del láser};
\end{scope}

%% BOMBEO
\draw[fill=verdecito] (M3) rectangle (M4);
\node[align=center,scale=0.5] at ($0.5*(M3)+0.5*(M4)$) {Fuente de energía exterior};
\foreach \z in {-9,-8,...,0,1,2,...,9}
    \draw[arrow] (-0.45*\z*\L/10,-0.04*\H) -- (-0.45*\z*\L/10,0.1*\H);

\end{tikzpicture}

\end{document}
