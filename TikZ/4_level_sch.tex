% !TeX program = lualatex
% Energy level diagram of a four-level pumping mechanism
% Autor: Ismael Torres García
\documentclass[tikz,border=3pt]{standalone}
\usetikzlibrary{calc, arrows.meta, decorations.pathmorphing, decorations.markings}
\usepackage[math-style=ISO,bold-style=ISO]{unicode-math}
\setmainfont[Renderer=OpenType]{Minion Pro}          %% Fuente comercial MinionPro de Adobe para el texto
\setmathfont[                                        %% Fuente MinionMath de Johannes Küsner para las matemáticas
    Extension = .otf,
    Path = /Users/ytoga/Library/Fonts/,
    Scale = 1,
    Script = Math,
    SizeFeatures = {
        {Size = -6, Font = MinionMath-Tiny,
        Style = MathScriptScript},
        {Size = 6-8.4, Font = MinionMath-Capt,
        Style = MathScript},
        {Size = 8.4-, Font = MinionMath-Regular},
    },
]{MinionMath-Regular}
\setmathfont[                         
    Extension = .otf,
    Path = /Users/ytoga/Library/Fonts/,
    version = bold,
]{MinionMath-Bold}

%%%%%%%%%%%%%%%%%%%%%%%%%%
%%% COMIENZO DEL DOCUMENTO
%%%%%%%%%%%%%%%%%%%%%%%%%%
\begin{document}

\begin{tikzpicture}[
  niveles/.style={thick,black},
  estadosabs/.style={thick,red},
  arrow/.style={-{Stealth[length=2mm,width=1.5mm]},blue,shorten >= 3pt,shorten <= 2pt},
  bombeo/.style={arrow,thick},
  estim/.style={arrow,thick,decorate,decoration={snake,amplitude=2.0,post length=5}},
]

%% DEFINICIONES
\def\anchura{90pt}
\def\espaciado{2pt}
\def\num{10}
\coordinate (sublevel) at (0, 10pt);
\coordinate (sublevel2) at (125pt, 120pt);

%% NIVELES FUNDAMENTAL Y SUPERIOR
\coordinate (S00) at ($(sublevel)$);
\coordinate (S01) at ($18*(sublevel)$);

% Dibujo los niveles de energía inferior y superior del bombeo
\draw[niveles] (S00) -- +(\anchura, 0) node[right] {\footnotesize Estado fundamental};
\foreach \i in {1,...,\num}{
  \draw[estadosabs] ([yshift=-\i*\espaciado] S01) -- +(\anchura, 0);
}

%% NIVELES DE LA TRANSICIÓN LASER
\coordinate (T00) at (sublevel2);
\coordinate (T01) at ($(T00) - 4*(sublevel)$);

% Dibujo los niveles de energía
\draw[niveles] (T00) -- +(\anchura, 0) node[right] {$|2\rangle$ (metaestable)};
\draw[niveles] (T01) -- +(\anchura, 0) node[right] {$|1\rangle$};

% Dibujo el bombeo hasta los estados de absorción
\draw[bombeo] ([xshift=.5*\anchura] S00) -- ([xshift=.5*\anchura,yshift=-.05*\anchura] S01);

% Dibujo el decaimiento rápido del nivel de absorción al estado metaestable
\draw[bombeo] ([xshift=.55*\anchura,yshift=-.1*\anchura] S01) -- ([xshift=.5*\anchura] T00);

% Dibujo la transición láser 
\draw[estim] ([xshift=.5*\anchura,yshift=-.05*\anchura] T00) -- ([xshift=.5*\anchura] T01);

% Dibujo el decaimiento rápido del nivel inferior al estado fundamental
\draw[bombeo] ([xshift=.5*\anchura] T01) -- ([xshift=.5*\anchura] S00);

%% TEXTO
\node at ($(S01)+(.5*\anchura,.1*\anchura)$) {\footnotesize Estados de absorción};
\node[rotate=-23] at ($.4*([xshift=.5*\anchura] S01)+.62*([xshift=.5*\anchura] T00)$) {\footnotesize Decaimiento rápido};
\node[align=center] at ($.5*([xshift=.8*\anchura] S00)+.5*([xshift=.8*\anchura] S01)$) {\footnotesize Transición \\\footnotesize por bombeo};
\node[align=center] at ($.5*([xshift=.8*\anchura] T00)+.5*([xshift=.8*\anchura] T01)$) {\footnotesize Transición \\\footnotesize láser};
\node[rotate=29] at ($.6*([xshift=.45*\anchura] T01)+.1*([xshift=.5*\anchura] S00)$) {\footnotesize Decaimiento rápido};
\end{tikzpicture}

\end{document}
