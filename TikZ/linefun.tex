% !TeX program = lualatex
% Lorentzian and Doppler line shape functions for absorption and emission of light
% Autor: Ismael Torres García
\documentclass[tikz,border=3pt]{standalone}
\usetikzlibrary{calc, arrows.meta}
\usepackage[math-style=ISO,bold-style=ISO]{unicode-math}
\setmainfont[Renderer=OpenType]{Minion Pro}          %% Fuente comercial MinionPro de Adobe para el texto
\setmathfont[                                        %% Fuente MinionMath de Johannes Küsner para las matemáticas
    Extension = .otf,
    Path = /Users/ytoga/Library/Fonts/,
    Scale = 1,
    Script = Math,
    SizeFeatures = {
        {Size = -6, Font = MinionMath-Tiny,
        Style = MathScriptScript},
        {Size = 6-8.4, Font = MinionMath-Capt,
        Style = MathScript},
        {Size = 8.4-, Font = MinionMath-Regular},
    },
]{MinionMath-Regular}
\setmathfont[                         
    Extension = .otf,
    Path = /Users/ytoga/Library/Fonts/,
    version = bold,
]{MinionMath-Bold}

\usepackage{pgfplots}
\pgfplotsset{compat=1.9}
\usepackage{siunitx}

% FUNCIONES
\pgfmathdeclarefunction{lorentz}{2}{%
  \pgfmathparse{(2/(#2*pi))/(1 + 4*(#1/#2)^2)}%
}
\pgfmathdeclarefunction{doppler}{2}{%
  \pgfmathparse{(2/#2)*(ln(2)/pi)^0.5*exp(-4*ln(2)*(#1/#2)^2)}%
}
%%%%%%%%%%%%%%%%%%%%%%%%%%
%%% COMIENZO DEL DOCUMENTO
%%%%%%%%%%%%%%%%%%%%%%%%%%
\begin{document}

\begin{tikzpicture} 

%% DEFINICIONES
\def\fwhmL{3} 
\def\fwhmD{5} 
\def\pmin{0}
\def\fmax{10}
\def\pmax{0.25}
\def\fnegL{-(\fwhmL^2/(lorentz(0,\fwhmL)*pi*\fwhmL)-0.25*\fwhmL^2)^0.5}
\def\fposL{(\fwhmL^2/(lorentz(0,\fwhmL)*pi*\fwhmL)-0.25*\fwhmL^2)^0.5}
\def\fnegD{-(-0.25*\fwhmD^2/ln(2)*ln(0.25*\fwhmD*doppler(0,\fwhmD)*(pi/ln(2))^0.5))^0.5}
\def\fposD{(-0.25*\fwhmD^2/ln(2)*ln(0.25*\fwhmD*doppler(0,\fwhmD)*(pi/ln(2))^0.5))^0.5}

%%%%%%%%%%%%%%%%
%%% DIBUJO 1 %%%
%%%%%%%%%%%%%%%%
%\begin{axis}[
%  every axis plot post/.append style={
%  mark=none,samples=100, smooth},
%  enlargelimits=upper,
%  ymax=1.05*\pmax, ymin=\pmin, 
%  xmax=1.4*\fmax, xmin=-1.4*\fmax, 
%  xlabel=$\nu(\unit{Hz})$, 
%  ylabel=$f_{L}(\nu)$,
%  axis lines=middle,
%  axis line style={thick, -Latex[round]},
%  xticklabels=\empty,
%  yticklabel=\empty,
%  every axis x label/.style={at={(current axis.right of origin)}, anchor=north west},
%  every axis y label/.style={at={(current axis.above origin)}, anchor=south east}
%]
%
%%% REPRESENTACIÓN DEL PERFIL DE LORENTZ 
%\addplot[blue, thick, domain=-1.3*\fmax:1.3*\fmax] {lorentz(x,\fwhmL)};   
%
%%% TEXTO
%\addplot[gray, densely dashed] coordinates {(0,{lorentz(0,\fwhmL)}) (-0.5*\fmax,{lorentz(0,\fwhmL)})} node[left] {\color{black}$f_{max}$}; 
%\addplot[gray, densely dashed] coordinates {(\fposL,{0.5*lorentz(0,\fwhmL)}) (-0.5*\fmax,{0.5*lorentz(0,\fwhmL)})} node[left] {\color{black}$f_{max}/2$}; 
%\addplot[gray, densely dashed] coordinates {(\fnegL,{0.5*lorentz(0,\fwhmL)}) (5,0.15)} node[left] {}; 
%\addplot[gray, densely dashed] coordinates {(\fposL,{0.5*lorentz(0,\fwhmL)}) (5+\fwhmL,0.15)} node[left] {}; 
%\addplot[thick] coordinates {(5,0.15) (5+\fwhmL,0.15)} node[right=10pt,above=0.1pt] {$\Delta \nu_{L}$ (FWHM)}; 

%%%%%%%%%%%%%%%%%
%%%% DIBUJO 2 %%%
%%%%%%%%%%%%%%%%%
\begin{axis}[
  every axis plot post/.append style={
  mark=none,samples=100, smooth},
  enlargelimits=upper,
  clip=false,
  ymax=0.9*\pmax, ymin=\pmin, 
  xmax=1.4*\fmax, xmin=-1.4*\fmax, 
  xlabel=$\nu(\unit{Hz})$, 
  ylabel=$f_{D}(\nu)$,
  axis lines=middle,
  axis line style={thick, -Latex[round]},
  xticklabels=\empty,
  yticklabel=\empty,
  every axis x label/.style={at={(current axis.right of origin)}, anchor=north west},
  every axis y label/.style={at={(current axis.above origin)}, anchor=south east}
]

%% REPRESENTACIÓN DEL PERFIL DOPPLER
\addplot[red, thick, domain=-1.3*\fmax:1.3*\fmax] {doppler(x,\fwhmD)};   

%% TEXTO
\addplot[gray, densely dashed] coordinates {(0,{doppler(0,\fwhmD)}) (-0.5*\fmax,{doppler(0,\fwhmD)})} node[left] {\color{black}$f_{max}$}; 
\addplot[gray, densely dashed] coordinates {(\fposD,{0.5*doppler(0,\fwhmD)}) (-0.5*\fmax,{0.5*doppler(0,\fwhmD)})} node[left] {\color{black}$f_{max}/2$}; 
\addplot[gray, densely dashed] coordinates {(\fnegD,{0.5*doppler(0,\fwhmD)}) (5,0.15)} node[left] {}; 
\addplot[gray, densely dashed] coordinates {(\fposD,{0.5*doppler(0,\fwhmD)}) (5+\fwhmD,0.15)} node[left] {}; 
\addplot[thick] coordinates {(5,0.15) (5+\fwhmD,0.15)} node[right=4pt,above=0.1pt] {$\Delta \nu_{D}$ (FWHM)}; 

\end{axis}

\end{tikzpicture}

\end{document}
