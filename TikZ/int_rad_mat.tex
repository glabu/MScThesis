% !TeX program = lualatex
% Light-matter interaction: absorption, spontaneous emission and stimulated emission of radiation 
% Autor: Ismael Torres García
\documentclass[tikz,border=3pt]{standalone}
\usetikzlibrary{calc, arrows.meta, decorations.pathmorphing, decorations.markings}
\usepackage[math-style=ISO,bold-style=ISO]{unicode-math}
\setmainfont[Renderer=OpenType]{Minion Pro}          %% Fuente comercial MinionPro de Adobe para el texto
\setmathfont[                                        %% Fuente MinionMath de Johannes Küsner para las matemáticas
    Extension = .otf,
    Path = /Users/ytoga/Library/Fonts/,
    Scale = 1,
    Script = Math,
    SizeFeatures = {
        {Size = -6, Font = MinionMath-Tiny,
        Style = MathScriptScript},
        {Size = 6-8.4, Font = MinionMath-Capt,
        Style = MathScript},
        {Size = 8.4-, Font = MinionMath-Regular},
    },
]{MinionMath-Regular}
\setmathfont[                         
    Extension = .otf,
    Path = /Users/ytoga/Library/Fonts/,
    version = bold,
]{MinionMath-Bold}

\usepackage{xcolor}
\definecolor{elc}{RGB}{204,204,0}      % Color de los electrones
\definecolor{laseruv}{RGB}{76,0,155}   % Color de los fotones

%%%%%%%%%%%%%%%%%%%%%%%%%%
%%% COMIENZO DEL DOCUMENTO
%%%%%%%%%%%%%%%%%%%%%%%%%%
\begin{document}

\begin{tikzpicture}[
  arrow/.style={-{Stealth[length=2mm,width=1.5mm]},shorten >=1pt},
  electron/.style={circle,fill=elc,draw=black,label=north east:$e^-$},
  hueco/.style={circle,fill=elc!20,draw=black,densely dashed},
  fotones/.style={arrow,laseruv,thick,decorate,decoration={snake,amplitude=3,segment length=8,post length=7}},
]

%% ABSORCIÓN
\draw[-, thick] (0,0) node[left] {$E_1$} -- (3,0) node[electron, midway] (a1) {} node[right] {$|1\rangle$};
\draw[-, thick] (0,1.5) node[left] {$E_2$} -- (3,1.5) node[hueco, midway] (a2) {} node[right] {$|2\rangle$};
\draw[arrow, thick] (a1) -- (a2);
\draw[fotones] (0.25,0.75) -- node[above] {$\gamma$} (1.25,0.75);
\node[below=5pt] at (a1) {\footnotesize Absorción};

%% EMISIÓN ESPONTÁNEA
\draw[-, thick] (4.3,0) node[left] {$E_1$} -- (7.3,0) node[hueco, midway] (e1) {} node[right] {$|1\rangle$};
\draw[-, thick] (4.3,1.5) node[left] {$E_2$} -- (7.3,1.5) node[electron, midway] (e2) {} node[right] {$|2\rangle$};
\draw[arrow, thick] (e2) -- (e1);
\draw[fotones] (6,0.75) -- node[above] {$\gamma$} (7,0.75);
\node[below=5pt] at (e1) {\footnotesize Emisión espontánea};

%% EMISIÓN ESTIMULADA
\draw[-, thick] (8.6,0) node[left] {$E_1$} -- (11.6,0) node[hueco, midway] (es1) {} node[right] {$|1\rangle$};
\draw[-, thick] (8.6,1.5) node[left] {$E_2$} -- (11.6,1.5) node[electron, midway] (es2) {} node[right] {$|2\rangle$};
\draw[arrow, thick] (es2) -- (es1);
\draw[fotones] (8.8,0.75) -- node[above] {$\gamma$} (9.8,0.75);
\draw[fotones] (10.3,0.55) -- (11.3,0.55) node[right=0.2pt] {$\gamma$};
\draw[fotones] (10.3,1) -- (11.3,1) node[right=0.2pt] {$\gamma$};
\node[below=5pt] at (es1) {\footnotesize Emisión estimulada};

\end{tikzpicture}

\end{document}
