% !TeX program = lualatex
% Work Breakdown Structure of the Master's Thesis Proyect
% Autor: Ismael Torres García
\documentclass[tikz,border=3pt]{standalone}
\usetikzlibrary{arrows.meta,shapes,positioning,shadows,trees}
\usepackage[math-style=ISO,bold-style=ISO]{unicode-math}
\setmainfont[Renderer=OpenType]{Minion Pro}          %% Fuente comercial MinionPro de Adobe para el texto
\setmathfont[                                        %% Fuente MinionMath de Johannes Küsner para las matemáticas
    Extension = .otf,
    Path = /Users/ytoga/Library/Fonts/,
    Scale = 1,
    Script = Math,
    SizeFeatures = {
        {Size = -6, Font = MinionMath-Tiny,
        Style = MathScriptScript},
        {Size = 6-8.4, Font = MinionMath-Capt,
        Style = MathScript},
        {Size = 8.4-, Font = MinionMath-Regular},
    },
]{MinionMath-Regular}
\setmathfont[                         
    Extension = .otf,
    Path = /Users/ytoga/Library/Fonts/,
    version = bold,
]{MinionMath-Bold}

\usepackage[edges]{forest}

\usepackage{xcolor}
\definecolor{miazul}{RGB}{0,30,155}
\definecolor{mimorado}{RGB}{140,45,165}
\colorlet{miazull}{white!80!miazul}
\colorlet{mimoradoo}{white!60!mimorado}

\tikzset{
  basic/.style = {
    draw, thin, align=center, 
    drop shadow
    },
  parent/.style = {
    basic, rounded corners=2pt, 
    text width=3.5cm,
   fill=miazul!50},
  child/.style = {
    basic, rounded corners=6pt,
    fill=miazull, text width=7em},
  grandchild/.style = {
    basic, 
    align=left, 
    fill=mimoradoo, text width=9em},
}

%%%%%%%%%%%%%%%%%%%%%%%%%%
%%% COMIENZO DEL DOCUMENTO
%%%%%%%%%%%%%%%%%%%%%%%%%%
\begin{document}

%%%%%%%%%%%%%%%%
%%%  DIBUJO  %%%
%%%%%%%%%%%%%%%%
\begin{forest} 
  for tree = {
    where level = 0{
      parent,
    }{
      folder,
      grow' = 0,
    },
    where level = 1{
      child,
      edge path' = {(!u.parent anchor) -- ++(0,-0.65cm) -| (.child anchor)},
      for descendants = {
        grandchild,
      },
    }{},
}
[Trabajo Fin de Máster, 
    [Documentación, 
      [\footnotesize Comprensión de los objetivos del TFM]
      [\footnotesize Obtención de la bibliografía y artículos]
      [\footnotesize Iniciación al código Dagon y aprendizaje]
      [\footnotesize Preparación del entorno de trabajo virtual]
    ]
    [Estudio, 
      [\footnotesize Asimilación de los principios básicos del trabajo]
      [\footnotesize Ejecución de las primeras simulaciones]
      [\footnotesize Escritura de los primeros programas y modificaciones]
    ]
    [Simulaciones, 
      [\footnotesize Organización del cuerpo de la memoria]
      [\footnotesize Continuación con las simulaciones intermedias]
      [\footnotesize Preparación de la forma y estructura de la memoria]
    ]
    [Escritura, 
      [\footnotesize Finalización del proceso de simulaciones]
      [\footnotesize Redacción de la memoria]
      [\footnotesize Revisión y corrección de la memoria preliminar]
      [\footnotesize Maquetación y entrega de la memoria final]
    ]
]
\end{forest}

\end{document}
