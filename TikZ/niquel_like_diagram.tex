% !TeX program = lualatex
% Energy levels of a niquel-like simplified diagram for a laser transition
% Autor: Ismael Torres García
\documentclass[tikz,border=3pt]{standalone}
\usetikzlibrary{calc, arrows.meta, decorations.pathmorphing, decorations.markings}
\usepackage[math-style=ISO,bold-style=ISO]{unicode-math}
\setmainfont[Renderer=OpenType]{Minion Pro}          %% Fuente comercial MinionPro de Adobe para el texto
\setmathfont[                                        %% Fuente MinionMath de Johannes Küsner para las matemáticas
    Extension = .otf,
    Path = /Users/ytoga/Library/Fonts/,
    Scale = 1,
    Script = Math,
    SizeFeatures = {
        {Size = -6, Font = MinionMath-Tiny,
        Style = MathScriptScript},
        {Size = 6-8.4, Font = MinionMath-Capt,
        Style = MathScript},
        {Size = 8.4-, Font = MinionMath-Regular},
    },
]{MinionMath-Regular}
\setmathfont[                         
    Extension = .otf,
    Path = /Users/ytoga/Library/Fonts/,
    version = bold,
]{MinionMath-Bold}

\usepackage{xcolor}

%%%%%%%%%%%%%%%%%%%%%%%%%%
%%% COMIENZO DEL DOCUMENTO
%%%%%%%%%%%%%%%%%%%%%%%%%%
\begin{document}

\begin{tikzpicture}[
  arrow/.style={-{Stealth[length=2mm,width=1.5mm]},shorten >= 2pt, blue},
  estim/.style={arrow,thick,decorate,decoration={snake,amplitude=1.5,post length=5}},
  espont/.style={arrow,densely dashed,decorate,decoration={snake,amplitude=1.5,post length=5}},
]

%% DEFINICIONES
\def\anchura{90pt}
\coordinate (sublevel) at (0pt, 10pt);
\coordinate (sublevel2) at (125pt, 155pt);
\coordinate (sublevel3) at (250pt, 105pt);

%% NIVELES FUNDAMENTAL Y SUPERIOR
\coordinate (S00) at ($-1*(sublevel)$);

% Dibujo los niveles de energía 
\draw[thick] ([xshift=125pt] S00) -- +(215pt, 0) node[midway, below=3pt] {\footnotesize $\mathrm{3d^{10}} (J=0)$};

%% NIVEL SUPERIOR DE LA TRANSICIÓN LASER
\coordinate (T00) at (sublevel2);

% Dibujo los niveles de energía
\draw[thick] (T00) node[left=0pt] {\scriptsize $(\mathrm{3d_{3/2},4d_{3/2}}) J=0$} -- +(\anchura, 0) 
node[midway, above=3pt] {\footnotesize $\mathrm{3d^94d}$};

% Dibujo el bombeo colisional
\draw[arrow] ([xshift=135pt] S00) -- ([xshift=10pt] T00) node[black, midway, left, align=center] {\footnotesize Excitación\\\footnotesize colisional};

%% NIVELES INFERIORES DE LA TRANSICIÓN LASER
\coordinate (U00) at (sublevel3);
\coordinate (U01) at ($(U00) - 2*(sublevel)$);

% Dibujo los niveles de energía
\draw[thick] (U00) node[left=0pt] {\scriptsize $(\mathrm{3d_{5/2},4p_{3/2}}) J=1$} -- +(\anchura,0) node[right=-20pt, above=2pt] {\footnotesize $\mathrm{3d^94p}$};
\draw[thick] (U01) node[left=0pt] {\scriptsize $(\mathrm{3d_{3/2},4p_{1/2}}) J=1$} -- +(\anchura,0);

% Dibujo las transiciones espontáneas
\draw[espont] ([xshift=75pt] U00) -- ([xshift=325pt] S00);
\draw[espont] ([xshift=60pt] U01) -- ([xshift=310pt] S00) node[black, midway, left, align=center] {\footnotesize Emisión\\\footnotesize espontánea};

% Dibujo las transiciones estimuladas
\draw[estim] ([xshift=80pt] T00) -- ([xshift=50pt] U00);
\draw[estim] ([xshift=45pt] T00) -- ([xshift=50pt] U01) node[black, midway, above=1pt, rotate=-28] {\footnotesize Transición láser (XUV)}; 

\end{tikzpicture}

\end{document}
