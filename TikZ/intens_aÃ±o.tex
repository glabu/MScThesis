% !TeX program = lualatex
% Laser intensity vs years adapted from "Optics in the relativistic regime
% Autor: Ismael Torres García
\documentclass[tikz,border=3pt]{standalone}
\usetikzlibrary{calc, arrows.meta, shapes.geometric}
\usepackage[math-style=ISO,bold-style=ISO]{unicode-math}
\setmainfont[Renderer=OpenType]{Minion Pro}          %% Fuente comercial MinionPro de Adobe para el texto
\setmathfont[                                        %% Fuente MinionMath de Johannes Küsner para las matemáticas
    Extension = .otf,
    Path = /Users/ytoga/Library/Fonts/,
    Scale = 1,
    Script = Math,
    SizeFeatures = {
        {Size = -6, Font = MinionMath-Tiny,
        Style = MathScriptScript},
        {Size = 6-8.4, Font = MinionMath-Capt,
        Style = MathScript},
        {Size = 8.4-, Font = MinionMath-Regular},
    },
]{MinionMath-Regular}
\setmathfont[                         
    Extension = .otf,
    Path = /Users/ytoga/Library/Fonts/,
    version = bold,
]{MinionMath-Bold}

\usepackage{siunitx}
\usepackage{xcolor}
\colorlet{mimorado}{purple!60!blue}
\colorlet{miazul}{blue!60}
\colorlet{miverde}{green!60}
\colorlet{miamarillo}{yellow!60}
\colorlet{minaranja}{orange!60}

%%%%%%%%%%%%%%%%%%%%%%%%%%
%%% COMIENZO DEL DOCUMENTO
%%%%%%%%%%%%%%%%%%%%%%%%%%
\begin{document}
\tikzset{
arrow/.style = {-{Triangle[angle'=30]}},
}

\begin{tikzpicture} 

%% DEFINICIONES
\def\ancho{4}
\def\largo{6}
\def\ymax{1.4}
\def\xmax{1.6}
\def\tick#1#2{\draw (#1) ++ (#2:0.1) --++ (#2-180:0.2)}

%% COORDENADAS
\coordinate (M) at (0.05,0.05);
\coordinate (Maux) at (0.4,0.4);
\coordinate (V1) at (0.05*\ancho,0.1*\largo);
\coordinate (V2) at (0.1*\ancho,0.3*\largo);
\coordinate (V3) at (0.5*\ancho,0.35*\largo);
\coordinate (V4) at (0.7*\ancho,0.52*\largo);
\coordinate (V5) at (1.21*\ancho,0.75*\largo);
\coordinate (V6) at (1.17*\ancho,0.85*\largo);
\coordinate (V7) at ($0.4*(V6)-0.4*(V4)$);

%% EJES
\draw[arrow] (0,0) -- (\xmax*\ancho,0) node[midway, below=0.5cm] {\footnotesize Años};
\draw[arrow] (0,0) -- (0,\ymax*\largo) node[midway, xshift=-1.0cm, rotate=90] {\footnotesize Intensidad enfocada (\unit{\W\per\cm^2})};
\draw[arrow] (0.9*\xmax*\ancho,0) -- (0.9*\xmax*\ancho,\ymax*\largo) node[midway, xshift=1.0cm, rotate=90] {\footnotesize Energía de los electrones (\unit{\eV})};
\draw[dashed] (0.9*\xmax*\ancho,0.2*\largo) -- (0.25*\xmax*\ancho,0.2*\largo);

\foreach \x / \year in {0/1960,0.2*\ancho/1970,0.4*\ancho/1980,0.6*\ancho/1990,0.8*\ancho/2000,\ancho/2010, 1.2*\ancho/2020, 1.4*\ancho/2030}
    \tick{\x,0}{90} node[below] {\scriptsize $\year$};
\foreach \y / \intensity in {0.1*\largo/10^{10},0.35*\largo/10^{15},0.6*\largo/10^{20},0.85*\largo/10^{25},1.1*\largo/10^{30},1.35*\largo/10^{35}}
    \tick{0,\y}{0} node[left] {\scriptsize $\intensity$};
\foreach \z / \energy in {0.2*\largo/1,0.53*\largo/10^{6},0.8*\largo/10^{12},1.05*\largo/10^{15}}
    \tick{0.9*\xmax*\ancho,\z}{0} node[right=5pt] {\scriptsize $\energy$};

%% REPRESENTACIÓN DEL FONDO
% Rectángulos
\begin{scope}[opacity=0.2]
    \fill[minaranja] (0,0) rectangle (0.9*\xmax*\ancho,0.38*\largo);
    \fill[miamarillo] (0,0.38*\largo) rectangle (0.9*\xmax*\ancho,0.53*\largo);
    \fill[miverde] (0,0.53*\largo) rectangle (0.9*\xmax*\ancho,0.8*\largo);
    \fill[miazul] (0,0.8*\largo) rectangle (0.9*\xmax*\ancho,1.05*\largo);
    \fill[mimorado] (0,1.05*\largo) rectangle (0.9*\xmax*\ancho,1.35*\largo);
\end{scope}

%% CURVA 
\draw[thick] (M) -- (V1) -- (V2) -- (V3) -- (V4) -- ($(V4)+.4*(V6)-.4*(V4)$);
\draw[thick,densely dotted,gray] ($(V4)+.4*(V6)-.4*(V4)$) -- (V6);
\draw[thick,densely dotted,gray] (V6) --++ (V7);
\draw[thick] ($(V4)+.4*(V6)-.4*(V4)$) to [out=45,in=180] (V5);

%% MARCAS
% Maiman
\fill[blue] (M) circle [radius=1.1pt];
\draw[blue] (M) to [out=0,in=180] (Maux) node[scale=0.5,right] {Maiman};

% Q-switch y mode-locking
\coordinate (QS) at ($(V1)+.5*(V2)-.5*(V1)$);
\coordinate (ML) at ($(V1)+.15*(V2)-.15*(V1)$);
\fill[red] (QS) circle [radius=1.1pt] node[right=1pt,scale=0.5] {Mode-locking} -- (1,1);
\fill[red] (ML) circle [radius=1.1pt] node[right=1pt,scale=0.5] {Q-switching};

% CPA
\coordinate (CPA) at (V3);
\fill[red] (CPA) circle [radius=1.1pt] node[right=1pt,scale=0.5] {CPA};

% OPCPA
\coordinate (OPCPA) at (V4);
\fill[red] (OPCPA) circle [radius=1.1pt] node[right=1pt,scale=0.5] {OPCPA};

% Récord 2021 CoReLS
\node[fill=blue,isosceles triangle,isosceles triangle apex angle=60,scale=0.2] at (V5) {};
\node[scale=0.5,blue,align=center,below=2pt] at (V5) {CoReLS \\ (récord actual)};

% CUOS
\coordinate (CUOS) at ($0.8*(V6)-0.8*(V4)$);
\fill[blue] ($(V4)+.4*(V6)-.4*(V4)$) circle [radius=1.1pt] node[right=1pt,scale=0.5] {CUOS};

% ILE
\coordinate (ILE) at ($0.8*(V6)-0.8*(V4)$);
\fill[gray] ($(V4)+.8*(V6)-.8*(V4)$) circle [radius=1.1pt] node[right=1pt,scale=0.5] {ILE};

% ELI
\coordinate (ELI) at (V6);
\fill[gray] (ELI) circle [radius=1.1pt] node[right=1pt,scale=0.5] {ELI};

%% ESCALAS DE FUNCIONAMIENTO
\node[scale=0.5,align=center] at (0.8*\ancho,0.25*\largo) {Átomos: $E_q=h\nu$ \\ (Materia ordinaria)};
\node[scale=0.5,align=center] at (0.3*\ancho,0.45*\largo) {Electrones ligados: $E_q=e^2/a_0$ \\ (Plasmas)};
\node[scale=0.5,align=center] at (0.35*\ancho,0.65*\largo) {Óptica relativista: $E_q=m_0c^2$ \\ (Plasmas relativistas, $v_{osc} \sim c$)};
\node[scale=0.35,align=center] at (0.4*\ancho,0.83*\largo) {Intensidad límite láser: $I=h\nu^3/c^2$ \\ (por \unit{\cm^2} de medio activo)};
\node[scale=0.5,align=center] at (0.4*\ancho,0.92*\largo) {Óptica ultrarelativista: $E_q=m_pc^2$ \\ (Plasmas cuánticos)};
\node[scale=0.5,align=center] at (0.45*\ancho,1.2*\largo) {Electrodinámica cuántica: $E_q=2m_0c^2$ \\ (Formación de pares del vacío)};
\node[scale=0.5] at (0.45*\xmax*\ancho,1.37*\largo) {Cromodinámica cuántica (Plasmas de quarks y gluones)};
\node[scale=0.5,above=3pt,rotate=45] at ($(V4)+.8*(V6)-.8*(V4)$) {Predicciones};

\end{tikzpicture}

\end{document}
