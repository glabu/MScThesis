% !TeX program = lualatex
% Lambert-Beer law for lasers penetrating an amplifier
% Autor: Ismael Torres García
\documentclass[tikz,border=3pt]{standalone}
\usetikzlibrary{calc, arrows.meta, fadings, decorations.markings}
\usepackage[math-style=ISO,bold-style=ISO]{unicode-math}
\setmainfont[Renderer=OpenType]{Minion Pro}          %% Fuente comercial MinionPro de Adobe para el texto
\setmathfont[                                        %% Fuente MinionMath de Johannes Küsner para las matemáticas
    Extension = .otf,
    Path = /Users/ytoga/Library/Fonts/,
    Scale = 1,
    Script = Math,
    SizeFeatures = {
        {Size = -6, Font = MinionMath-Tiny,
        Style = MathScriptScript},
        {Size = 6-8.4, Font = MinionMath-Capt,
        Style = MathScript},
        {Size = 8.4-, Font = MinionMath-Regular},
    },
]{MinionMath-Regular}
\setmathfont[                         
    Extension = .otf,
    Path = /Users/ytoga/Library/Fonts/,
    version = bold,
]{MinionMath-Bold}

\usepackage{xcolor}
\definecolor{atoms}{RGB}{51,153,255}  % Color de los átomos
\definecolor{laser}{RGB}{76,0,155}    % Color del láser

%% DECORACIÓN DEL LÁSER  
\makeatletter
\pgfkeys{/pgf/decoration/.cd,
    start color/.store in = \colorinicial,
    end color/.store in   = \colorfinal
}%

\pgfdeclaredecoration{cambio color}{inicio}{

% Estado inicial
\state{inicio}[%
    width                     = 0pt,
    next state                = linea,
    persistent precomputation = {\pgfmathdivide{50}{\pgfdecoratedpathlength}\let\increment=\pgfmathresult\def\x{0}}]%
{}%

% Estado de la línea
\state{linea}[%
    width                      = 0.5pt,
    persistent postcomputation = {\pgfmathadd@{\x}{\increment}\let\x=\pgfmathresult}
]%
{%
  \pgfsetlinewidth{\pgflinewidth}%
  \pgfsetarrows{-}%
  \pgfpathmoveto{\pgfpointorigin}%
  \pgfpathlineto{\pgfqpoint{0.75pt}{0pt}}%
  \pgfsetstrokecolor{\colorfinal!\x!\colorinicial}% %% sentido de la difuminación
  \pgfusepath{stroke}%
}%

% Estado final
\state{final}{%
  \pgfsetlinewidth{\pgflinewidth}%
  \pgfpathmoveto{\pgfpointorigin}%
  \color{\colorfinal!\x!\colorinicial}%
  \pgfusepath{stroke}% 
}
}
\makeatother

%%%%%%%%%%%%%%%%%%%%%%%%%%
%%% COMIENZO DEL DOCUMENTO
%%%%%%%%%%%%%%%%%%%%%%%%%%
\begin{document}

%% DEFINICIONES 
\def\cols{60}                         % Número de filas
\def\rows{15}                         % Número de columnas
\def\SquareUnit{0.35}                 % Lado del cuadrado (cm)
\pgfmathsetmacro\RmaxParticle{0.13}   % Radio máximo de los átomos
\def\BeforeLight{5}                   % Distancia antes de los átomos

\begin{tikzpicture}[
    x          = \SquareUnit cm,
    y          = \SquareUnit cm,
    line width = 2pt
]

%% RECORRIDO  
% Antes de los átomos
\draw[laser!20!white, 
  decoration={markings, mark=at position 0.5 with \arrow[]{latex}}, postaction={decorate}] 
    (-\BeforeLight,{0.5*\rows*\SquareUnit}) --++ (\BeforeLight,0) 
    node[midway, above, black]{$I_0$};

% Durante los átomos
\draw[
    decoration={cambio color, start color=laser, end color=laser!20!white}, decorate] 
    (\cols*\SquareUnit,{0.5*\rows*\SquareUnit}) --++ (-\cols*\SquareUnit,0);

% Después de los átomos
\draw[laser, 
    decoration={markings, mark=at position 0.5 with \arrow[]{latex}}, postaction={decorate}] 
    ({\cols*\SquareUnit},{0.5*\rows*\SquareUnit}) --++ (\BeforeLight,0) 
    node[midway, above, black]{$I$};

%% CANAL AMPLIFICADOR 
\foreach \i in {1,...,\cols}{
	\foreach \j in {1,...,\rows}{
    \pgfmathsetmacro\radius{\RmaxParticle*rnd};
    \pgfmathsetmacro\l{\SquareUnit-2*\radius};
    \pgfmathsetmacro\x{(\i-1)*\SquareUnit+\radius+\l*rnd};
    \pgfmathsetmacro\y{(\j-1)*\SquareUnit+\radius+\l*rnd};
    \fill[atoms] (\x,\y) circle[radius=\radius];
  }
}

%% LONGITUD CANAL 
\draw[|->, >={Triangle[angle'=30]}, line width = 0.8pt, black] 
($(0,\rows*\SquareUnit)+(0,1)$)--++(0.25*\cols*\SquareUnit,0) 
node[midway, above]{$z$};

%% LAMBERT-BEER   
\node[anchor=north, inner sep = 1ex] at (current bounding box.south) {$I(z,\nu)=I_0\symrm{e}^{G(\nu)z}$};

\end{tikzpicture}

\end{document}
