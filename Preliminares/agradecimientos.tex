\addchap{Agradecimientos}
\lettrine{E}{ste año ha sido especialmente} complicado para mí, por diversos motivos, y en situaciones así hay muchas personas a las que me gustaría agradecer el apoyo y la ayuda recibida durante todos estos meses tan largos, y al mismo tiempo, tan difíciles. Por suerte, mi círculo más estrecho de amistades que me ha prestado ayuda durante este tiempo es pequeño, pero lleno de buenas personas, así que la tarea de reconocer todo el amor recibido no ha sido realmente complicada. 

En primer lugar, las personas más importantes de esta historia son mis padres, sin los que una inmensa mayoría de las dificultades que me he visto obligado a enfrentar durante mi vida podría haber superado. Gracias a ellos, he podido conseguir pequeños objetivos y logros que hubieran sido imposibles sin su soporte, además de permitirme tomar mis propias decisiones y seguir el camino que más se adaptase a mi forma de ser, aunque pueda ser una trayectoria dificultosa y, seguramente, muy extraña y absurda en muchos sentidos.

En segundo lugar, me gustaría agradecer a mis amigos la atención y el cariño que he recibido durante este año tan atípico para mí. A Javier, María, Miguel, Elena y Sara por hablar conmigo y escucharme cuando lo necesitaba, por la comprensión recibida durante tantos meses y por los pequeños ---y grandes--- detalles que hacen más agradable el día. A Carmen, por ser una persona maravillosa que me ha escuchado he intentado ayudar siempre que ha podido, por las conversaciones y los paseos tan amenos y agradables después de salir de las clases, y por querer mi felicidad y bienestar. A Irene, por animarme y darme todo el apoyo posible, a pesar de no haber podido hacerlo en persona tanto como me hubiese gustado.

Mi \enquote{consejo de sabios} físicos, Jorge y Rafael, siempre tienen discusiones muy interesantes sobre la situación de la ciencia y la investigación de los más jóvenes como nosotros, y como siempre, también me han ayudado a distraerme de la rutina diaria con sus temáticas de doctorado en teoría de cuerdas computacional y astrofísica, aunque no tenga ni la más remota idea de estos temas tan absolutamente fascinantes. 

Para continuar, quiero mencionar la importancia que han tenido mi psicóloga María Rubio García, Paola (que estaba de prácticas, y no conseguí averiguar su nombre completo), y mi psiquiatra el Dr. José Antonio Suárez Meneses, en tratarme cuando comenzó el insoportable, infernal y asfixiante episodio de depresión que he sufrido desde comienzos del curso. La dedicación y el trabajo necesario para indagar y averiguar detalles sobre mi vida que pudieran ayudarme a sobrellevar la situación y mejorar algunos aspectos del día a día ---de los que seguramente no era consciente---, es algo que nunca podré agradecer lo suficiente.

Nunca pensé que me vería envuelto en un problema así, pero cuando llegó el momento tuve la determinación necesaria y pude tomar la decisión de recurrir a la ayuda de profesionales para intentar superar algo que ---antes de empezar el tratamiento--- creía insuperable. Durante todo este tiempo he aprendido a darle un mayor equilibrio a mi vida, a enfrentar muchas situaciones de una forma diferente y, en general, a visualizar el futuro con una mirada distinta, más optimista y menos catastrófica, para ser menos exigente conmigo mismo y quitarle importancia a mis elucubraciones y pensamientos internos, que por lo general suelen ser una manifestación irracional de nuestras inseguridades y miedos, y no tanto de la realidad que vivimos.

Para terminar, no puedo olvidarme de mi tutor, Eduardo, que me ofreció a principios de curso repetir la posibilidad de seguir estudiando con este proyecto el interesante y espectacular mundo de la investigación dentro del inmenso campo de la Óptica y la Fotónica, concretamente el relacionado con los láseres y sus aplicaciones. Estoy seguro de que todos los profesionales investigadores que comenzaron teniendo interés y curiosidad por la vocación científica recuerdan a la persona que les introdujo por primera vez a la investigación, y en mi caso, termine o no dentro del mundo investigador, Eduardo ha sido la primera y única persona que me viene a la mente. 

Tristemente, en una universidad donde prácticamente nadie menciona la salida investigadora como una posibilidad, Eduardo me ha permitido acercarme más a ese universo tan desconocido e ignorado por los estudiantes de carreras técnicas como la ingeniería y la arquitectura, motivo más que suficiente para estar eternamente agradecido por ayudarme a distraerme del ---en demasiadas ocasiones--- aburrido y monótono trabajo de estudio que se había convertido en costumbre para mí durante estos últimos años.

A todas las personas que han estado a mi lado apoyándome en esos momentos de mayor necesidad, ¡muchas gracias de corazón!, y espero que con el tiempo os pueda devolver todo lo que habéis hecho por mí.


