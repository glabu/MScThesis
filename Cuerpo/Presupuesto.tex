\chapter{Presupuesto}\label{cap:10}
\lettrine{D}{urante la realización de este proyecto} han tenido lugar unos costes materiales y personales que pueden cuantificarse y clasificarse en dos grupos diferenciados. Es importante subrayar que, los resultados económicos presentados a continuación consisten en estimaciones aproximadas, basadas en los precios (en promedio) de los recursos empleados, puesto que no ha tenido una asignación presupuestaria o retribución económica asociada. 

Para comenzar, la división de los costes incurridos puede hacerse en directos e indirectos. El primer grupo comprende los gastos relativos a recursos materiales y humanos involucrados en la ejecución del trabajo, mientras que el segundo grupo consiste en gastos de naturaleza variable, principalmente los sistemas de alimentación eléctrica, servicio de conexión a internet y agua caliente sanitaria. Este último grupo está asociado íntegramente a la actividad personal del alumno en su domicilio particular, debido a que la totalidad del proyecto ha tenido lugar a través de un servidor remoto vinculado al Instituto de Fusión Nuclear \enquote{Guillermo Velarde} (IFN-GV). 

Dentro del primer bloque, es posible distinguir distintos tipos de gastos materiales, relacionados simplemente con la amortización de los ordenadores y el coste de las licencias de programas informáticos utilizados. El código Dagon encargado de las simulaciones numéricas (incluyendo todos los programas de Python, Fortran y Octave que participan en Dagon), y el programa VisIt para la obtención de determinadas imágenes, son software libre, por tanto no representan ningún gasto. De este modo, la única licencia a considerar es el paquete institucional de Microsoft $365$ A3 empleada por la Universidad Politécnica de Madrid, con un coste estimado de $100$ € anuales.

Los equipos empleados están formados por el ordenador personal del alumno y el servidor remoto ocupado de ejecutar las códigos numéricos. Teniendo en cuenta la considerable demanda computacional producida, es conveniente considerar un sistema de amortización lineal a $5$ años, extendiéndose a la duración total del trabajo académico (tres cuatrimestres) en ambos sistemas informáticos. El equipo remoto tiene un coste estimado de $4000$ €, mientras que el ordenador particular de $3000$ €. 

La Tabla \ref{tab:10.1} resume los costes directos debidos a los equipos y licencias.

\begin{table}[htpb]
  \centering
  \caption{Costes directos debidos a los recursos materiales.}
  \label{tab:10.1}
  \begin{tabular}{lS}
    \toprule
    Recursos materiales          & {Coste (€)} \\
    \midrule
    Licencia Dagon               & 0           \\
    Licencia VisIt               & 0           \\
    Licencia Microsoft $365$ A3  & 125         \\
    Amortización servidor        & 800         \\
    Amortización portátil        & 600         \\
    TOTAL                        & 1525        \\
    \bottomrule
  \end{tabular}
\end{table}

La Tabla \ref{tab:10.2} muestra los costes directos de las personas que han participado en el proyecto, en este caso, las estimaciones de los salarios brutos del profesor tutor del proyecto (investigador senior, programa de investigación Ramón y Cajal) y el alumno encargado del trabajo. Para simplificar el cálculo del coste horario de las horas de trabajo laborables, es preferible escoger un $28$\% de aportación a la seguridad social sobre los salarios brutos. En la Tabla \ref{tab:10.3}, aparece el coste total de los recursos humanos correspondientes a las horas de dedicación durante los tres cuatrimestres de duración (incluyendo los periodos vacacionales).

\begin{table}[htpb]
  \centering
  \caption{Costes horarios del profesor tutor y el alumno.}
  \label{tab:10.2}
  \begin{tabular}{lSS}
    \toprule
                              & {Alumno} & {Tutor} \\
    \midrule
    Salario Bruto (€/año)     & 8240,5   & 40000   \\
    Seguridad Social (€/año)  & 2307,25  & 11200   \\
    Horas laborales (año)     & 1085     & 1736    \\
    Coste horario (€/hora)    & 9,75     & 29,5    \\
    \bottomrule
  \end{tabular}
\end{table}

\begin{table}[htpb]
  \centering
  \caption{Costes directos del profesor tutor y el alumno.}
  \label{tab:10.3}
  \begin{tabular}{lSSS}
    \toprule
               & {Dedicación (horas)} & {Coste horario (€/hora)} & {TOTAL (€)} \\
    \midrule
    Alumno     & 1356,25              & 9,75                     & 13225       \\
    Tutor      & 125                  & 29,5                     & 3687,5      \\
    TOTAL      & 1481,25              &                          & 16912,5     \\
    \bottomrule
  \end{tabular}
\end{table}

Finalmente, como última hipótesis, los costes indirectos incurridos pueden estimarse suponiendo una contribución del $15$\% sobre el total de los costes directos obtenidos. La Tabla \ref{tab:10.4} recoge los resultados finales, proporcionando un coste final del proyecto de $25655,78$ €.

\begin{table}[htpb]
  \centering
  \caption{Coste total del proyecto.}
  \label{tab:10.4}
  \begin{tabular}{lS}
    \toprule
    Costes totales      & {Coste (€)} \\
    \midrule
    Recursos materiales & 1525        \\
    Recursos humanos    & 16912,5     \\
    Indirectos ($15$\%) & 2765,625    \\
    Sin IVA             & 21203,125     \\
    IVA ($21$\%)        & 4462,65     \\
    TOTAL               & 25655,78    \\
    \bottomrule
  \end{tabular}
\end{table}







